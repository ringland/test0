% This is a small sample LaTeX input file (Version of 10 April 1994)
%
% Use this file as a model for making your own LaTeX input file.
% Everything to the right of a  %  is a remark to you and is ignored by LaTeX.

% The Local Guide tells how to run LaTeX.

% WARNING!  Do not type any of the following 10 characters except as directed:
%                &   $   #   %   _   {   }   ^   ~   \   

\documentclass[a4paper]{article}        % Your input file must contain these two lines 

\usepackage{amsmath}
\usepackage{amsfonts}
\usepackage{amssymb}
%\usepackage{fancyhdr}
\usepackage{graphicx}
\usepackage{tikz}
\usepackage{xcolor}
\usepackage{framed}
%\usepackage{mdframed}

\definecolor{shadecolor}{rgb}{.95,.95,.95}

\usepackage[urlcolor=blue, 
	colorlinks=true,
	pdfauthor={Adam Cunningham}]{hyperref}

\renewcommand*{\familydefault}{\sfdefault}

\pagestyle{empty}

\usepackage[
top    = 2.00cm,
bottom = 2.50cm,
left   = 2.00cm,
right  = 2.00cm]{geometry}

\setlength{\parindent}{0in}
%\setlength{\oddsidemargin}{0.25in}
%\setlength{\textwidth}{6in}
\setlength{\parskip}{.05in}
%\setlength{\voffset}{-.5in}
%\setlength{\textheight}{9.25in}

\definecolor{myblue}{rgb}{0,.25,.6}
\definecolor{light}{gray}{.80}

\newcommand{\greyline}[2]{\vspace{.1in}\colorbox{light}{\parbox{#2in}{\textbf{#1}}}}

\newcommand{\bigheading}[1]{\bigskip \colorbox{light}{\parbox{6.6in} {\color{myblue} \sf \textbf{\Large#1}}} \smallskip}
\newcommand{\heading}[1]{\medskip {\color{myblue} \sf \textbf{\Large#1}}}
\newcommand{\subheading}[1]{\medskip {\color{myblue} \sf \large#1}}
\newcommand{\subject}[1]{\smallskip \textbf{\large#1:}}
\newcommand{\ssheading}[1]{\medskip {\color{myblue} \sf \textbf{\normalsize#1}} \large}

\newcommand{\sumz}[1]{\sum\limits_{#1 = 0}^{\infty}}
\newcommand{\sumo}[1]{\sum\limits_{#1 = 1}^{\infty}}

\newcommand{\der}[2]{\frac{d#1}{d#2}}
\newcommand{\dertwo}[2]{\frac{d^2#1}{d#2^2}}
\newcommand{\pder}[2]{\frac{\partial#1}{\partial#2}}
\newcommand{\bigpder}[2]{\dfrac{\partial#1}{\partial#2}}
\newcommand{\pdertwo}[2]{\frac{\partial^2#1}{\partial#2^2}}
\newcommand{\bigpdertwo}[2]{\dfrac{\partial^2#1}{\partial#2^2}}

\setlength{\unitlength}{.5in}
\newcommand{\proof}{\textit{Proof. }}
\newcommand{\proofend}{\hfill QED}

\newcommand{\var}{\mathrm{Var}}
\newcommand{\cov}{\mathrm{Cov}}

% --------------------------------------------------------------------------------
% 
% --------------------------------------------------------------------------------

\newenvironment{say}{}{}

% --------------------------------------------------------------------------------
% Board enrironment is for stuff to write on the board
% --------------------------------------------------------------------------------

\newenvironment{board}
{\definecolor{shadecolor}{rgb}{.85,1,.85}
\begin{shaded*}}
{\end{shaded*}}

% --------------------------------------------------------------------------------
% Code environment is for code to run in IPython
% --------------------------------------------------------------------------------

\usepackage{listings}

\definecolor{codeblue}{rgb}{0,0,1}
\definecolor{codegreen}{rgb}{0,0.6,0}
\definecolor{codegray}{rgb}{0.5,0.5,0.5}
\definecolor{codeorange}{rgb}{1,.5,0}
\definecolor{codepurple}{rgb}{0.58,0,0.82}
\definecolor{codered}{rgb}{1,0,0}
\definecolor{backcolour}{rgb}{0.95,1,1}
 
\lstdefinestyle{mystyle}{
    aboveskip=1em,
    backgroundcolor=\color{backcolour}, 
    basicstyle=\large, 
    belowskip=1em, 
    breakatwhitespace=false,         
    breaklines=true, 
    captionpos=b,   
    commentstyle=\color{codered},
    frame=single, 
    keepspaces=true, 
    keywordstyle=\color{codeorange},
    language=Python,              
    numbers=left,                    
    numbersep=5pt, 
    numberstyle=\normalsize\color{codegray}, 
    rulecolor=\color{light},                  
    showspaces=false,                
    showstringspaces=false,
    showtabs=false, 
    stringstyle=\color{codegreen},           
    tabsize=2
}
 
\lstset{style=mystyle}

% --------------------------------------------------------------------------------
% Question environment is for questions to ask the class
% --------------------------------------------------------------------------------

\newenvironment{question}{\begin{framed}}{\end{framed}}

% --------------------------------------------------------------------------------
% Activity environment is for an activity for the class to do together
% --------------------------------------------------------------------------------

\newenvironment{activity}
{\begin{framed}
\textbf{Activity:}
\begin{shaded}
}
{\end{shaded}
\end{framed}
}

% --------------------------------------------------------------------------------
% finish environment is for the last thing to mention before the class finishes
% --------------------------------------------------------------------------------

\newenvironment{finish}
{\definecolor{shadecolor}{rgb}{1,.9,.9}
\begin{shaded*}}
{\end{shaded*}}

% --------------------------------------------------------------------------------
% quiz environment is for the quizzess
% --------------------------------------------------------------------------------

\newenvironment{quiz}
{\begin{framed}}
{\end{framed}}

% --------------------------------------------------------------------------------
% optional environment is for sections that could be missed out if needed
% --------------------------------------------------------------------------------

\newenvironment{optional}
{\begin{leftbar}}
{\end{leftbar}}

% --------------------------------------------------------------------------------

